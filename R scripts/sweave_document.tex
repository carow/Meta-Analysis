\documentclass[11pt, a4paper]{article} 

\usepackage[utf8]{inputenc}  

\usepackage{caption}  			% provides commands for handling caption sizes etc.
%\usepackage[a4paper, left=25mm, right=20mm, top=25mm, bottom=20mm]{geometry}		 % to easily change margin widths: https://www.sharelatex.com/learn/Page_size_and_margins

\usepackage{etoolbox}    % for conditional evaluations!
\usepackage[bottom]{footmisc}  % I love footnotes! And they should be down at the bottom of the page!
\usepackage{graphicx}        % when using figures and alike
\usepackage[hidelinks]{hyperref}		% for hyperreferences (links within the document: references, figures, tables, citations)

\usepackage{euler}     % a math font, only for equations and alike; call BEFORE changing the main font; alternatives: mathptmx, fourier, 
%\usepackage{gentium} % for a different font; you can also try: cantarell, charter, libertine, gentium, bera, ... http://tex.stackexchange.com/questions/59403/what-font-packages-are-installed-in-tex-live

%------------------------------------------------------------------------------------------------------
%------- text size settings --------------
\setlength{\textwidth}{16cm}% 
\setlength{\textheight}{25cm} %23 
%(these values were used to fill the page more fully and thus reduce the number of pages!)
\setlength{\topmargin}{-1.5cm} %0
\setlength{\footskip}{1cm} %
%\setlength{\hoffset}{0cm} %
\setlength{\oddsidemargin}{1cm}%
\setlength{\evensidemargin}{-.5cm}%
\setlength{\parskip}{0cm} % Abstand zwischen Absätzen
% ----------------------------------------------------------------
\renewcommand{\textfraction}{0.1} % allows more space to graphics in float
\renewcommand{\topfraction}{0.85}
%\renewcommand{\bottomfraction}{0.65}
\renewcommand{\floatpagefraction}{0.70}


\frenchspacing %http://texwelt.de/wissen/fragen/1154/was-ist-french-spacing-was-macht-frenchspacing
%------------------------------------------------------------------------------------------------------
%------------------------------------------------------------------------------------------------------

\usepackage{Sweave}
\begin{document}
\Sconcordance{concordance:sweave_document.tex:sweave_document.Rnw:%
1 37 1 1 0 1 1 1 7 12 1 1 45 2 1 1 7 5 1}


\title{Appendix}
\author{Torfinn Belbo \& Carolina Wackerhagen}

\date{\today}

\maketitle

We created an appendix of meta-analysis paper. To be able to visualize the output, we used an example dataset taken from Gibson et al. 2011.\\ 

The dataset to be working with should be named ``data.sub''. The conducted analysis using the function rma from the metafor pacakge should be renamed as such: rma of a random effects model should be named ``rma.RE'' and an rma of a fixed effects model should be named ``rma.FE'' in order for the automatisation to work. IF a meta-regression has been conducted, it should be called ``rma.RE.meta'' or ``rma.FE.meta'' respectively. Other than that, the metafor package in R needs to be installed.  



For creating an appendix with an unkown dataset... 


%use  an if expression. If rma.FE = TRUE, then do the plots for FE, if not, do the plots with RE in the code. 



\end{document}
